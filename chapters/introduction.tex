%!TEX root = ../main.tex

\chapter{Introduction}\markboth{INTRODUCTION}{} \markright{INTRODUCTION}\label{ch:introduction}

\chapterprecishere{``To explain culture, then, is to explain why and how some ideas happen to be contagious.''}{Dan Sperber}{Explaining Culture}

\vspace{3.5ex plus 1ex minus .2ex}
\noindent This study examines the mechanisms underlying the selection and transmission of stories from an evolutionary perspective. Stories, much like human beings, have lives, which start the moment a particular event or experience is formulated in narrative form. Their further `survival' crucially depends on their being re-told: a story that does not acquire new tellers ``can have no life beyond the life of the original person who experienced the events and first formulated them as a story''\autocite{linde:2009}. By being transmitted from one individual to another, stories may propagate, sometimes from generation to generation. Some stories propagate so effectively -- or, to use Sperber's wording, `contagiously' -- that they become part of a culture's central heritage. 

When stories are propagated, they tend to get altered and reshaped with each retelling. Experimental studies on story transmission have shown that factors such as memory capacity and other social-cognitive pressures can have severe impact on how accurately stories are transmitted.\autocite{bartlett,mesoudiwhiten:2010,breithaupt:2015} In other words, stories are commonly subjected to `imperfect replication', and, as such, their transmission leads to their gradual transformation. Interestingly, while such experimental studies merely regard the reshaping of stories as a natural by-product of their being transmitted by multiple agents, some critics have argued that reshaping or transformation is in fact an essential prerequisite for a story's survival; a story \emph{needs} to be recurrently adapted to secure its position in culture\autocite{stephens_mccallum,collins_ridgman}. As \citeauthor{geerts_bossche} state: ``Without all of the transformations, which keep the work available, it would no longer be read and therefore `die out'''.\autocite{geerts_bossche} The fact that stories need to be altered to survive even seems to be the case for fairy tales, some of which are among the most `contagious' stories in present-day Western culture (e.g.\ ``Cinderella'' and ``Snow White''). Notwithstanding the fact that many fairy tales have been kept in relatively stable form for centuries (at least on an abstract, structural level), the current study wishes to argue that even fairy tales undergo progressive alterations as the result of a process of modification in which changes are gradually accumulated.

Although the complex interplay between adaptation and selection in story transmission has been recognized by various scholars in literary and folklore studies\autocite[E.g.][]{boyd:2009,zipes:2006,zipes:2012,sanders:2006,hutcheon:2013}, these claims are, unfortunately, largely programmatic: the current accounts of adaptation and selection in story transmission posit thought-provoking theoretical claims, but they do not systematically nor quantitatively test these claims. Thus, if we wish to arrive at a more concrete and precise understanding of the selection forces that underlie the transmission of stories, we are faced with a challenge: we need to suggest and develop new ways to empirically and quantitatively verify such programmatic claims.

With the current study, I aim to tackle this challenge by drawing inspiration from the cultural evolution research program\autocite{sforzafeldman:1981,boyd_richerson:1985,mesoudi:2011}. This program aims to enhance our understanding of cultural selection processes by means of formal models of cultural evolution, which allow us to quantitatively assess changes in cultural variation over time. In recent years, special attention has been devoted to a neutral model of cultural evolution which is comparable to biological drift. In this neutral model, individuals either introduce new variants of a particular cultural trait to the population or copy existing variants proportionally to the frequency of these variants in the population, and more popular variants are more likely to get adopted than less popular ones. This simple model accurately predicts a variety of cultural evolutionary processes, such as the selection of keywords in academic publications\autocite{Bentley:2008}, the choice of baby names\autocite{Hahn:2003}, and the popularity of particular dog breeds\autocite{herzog:2004,ghirlanda:2014}. This random copying model, strikingly, appears to be so accurate that researchers consider it to be the null hypothesis in the description of cultural evolutionary processes. Put differently, it is held that cultural evolution proceeds randomly, unless proven otherwise. Formal models of cultural evolution have at least three major advantages over the programmatic statements that are often employed in literary and folklore studies, which are often based on intuition or research on a small corpus. First, to explain prevalent culture, formal models call for detailed and replicable descriptions and definitions of the mechanisms and processes that underlie cultural change. Second, by adopting a formal approach, researchers can attempt to isolate and systematically compare forces of selection. Finally, one of the most important characteristics of quantitative, computational models is that they yield specific predictions that can be tested experimentally or against observations in real-world data\autocite[For an elaborate discussion, see][]{mesoudi:2011}.

The predictions made by quantitative models, such as, for instance, those built by \citeauthor{sforzafeldman:1981}\autocite{sforzafeldman:1981}, have been supported by experimental studies of story transmission. In several of these studies, story transmission is experimentally simulated by means of a `transmission chain' method, which closely resembles the game called `Chinese whispers' or `telephone' in the United States.\autocite{bartlett,mesoudiwhiten:2008} In these transmission simulations, participants attempt to memorize a story from either reading or listening and pass on their version of the story to the next participant generation. An interesting example of such an experimental study is the one by \citeauthor{eriksson:2012} who describe an inhibiting effect of multiple `parents' in transmission chains (i.e.\ chains in which stories are retold on the basis of more than one previous version) on the rate at which stories change.\autocite{eriksson:2012} Although experimental studies such as these have yielded a wealth of insights about underlying mechanisms of story transmission, a striking lack in cross-disciplinary communication and exchange of information with literary and folklore research prevents these insights to find their way across fields. Additionally, experimental story transmission studies have focused primarily on finding explanations for the progressive alterations of stories, and not on determinants of binary selection\autocite[Notable exceptions include, e.g.][]{Heath:2001,eriksson:2014}. Put differently, while story transmission studies mainly seek to map out how one and the same story changes through transmission, they do not explain why a story — or in some cases a group or `assemblage' of stories — is recurrently retold and becomes successful, whereas others go extinct altogether.\autocite{zipes:2006,zipes:2012} As such, it remains unclear how experimental results should be interpreted in light of real-world story transmission and selection.

One of the main obstacles to study the phenomena observed in real-world story transmission processes is the lack of detailed historical data that specifies the exact paths of transmission between different individuals and generations. The task of identifying such paths of story transmission would be easier to accomplish with diachronic data that describes who learns or copies from whom\autocite{kandler:2015}. Unfortunately, although a number of folktale digitization initiatives which provide large-scale collections of folktales have been undertaken recently\autocite{abello:2012,barre:2012,meder:2010,meder:2016}, longitudinal data collections of real-world stories are still virtually non-existent. The current study aims to meet this demand by delivering a large diachronic collection of Dutch versions of the fairy tale ``Little Red Riding Hood''.

The central question of this study is whether story transmission can be understood and hence should be described as a cultural evolutionary process, and, if so, which evolutionary mechanisms can be identified in story transmission processes. The main focus of the study are the mechanisms that underlie story transmission processes in folktales. Additionally, the goal is to unite the study's findings, whether they are in line with or contrary to the predictions of evolutionary models, with existing literary and folklore accounts of story transmission and selection. As such, this study contributes to the further synthesis of the disciplines of cultural evolution, literature and folklore. The research's relevance ranges from providing answers to methodological questions regarding how and to what extent evolutionary mechanisms underlying real-world story transmission can be studied from real-world, historical corpus data, to offering more abstract, theoretical reflections on pre-textual relations between stories. In particular, I will address the following four research questions:
\begin{description}
\item[Research question 1:] How should stories be formally represented in order to study real-world story transmission and selection with computational means?
\item[Research question 2:] Can historical corpus data be employed to reveal content biases in story transmission?
\item[Research question 3:] Can we describe the diachronic development of stories as a gradual accumulation of modifications and hence as an evolutionary process?
\item[Research question 4:] Can story transmission and selection be described as a random selection process, and what other mechanisms underlying story transmission and selection can be discerned from historical corpus data?
\end{description}
Research question 1 serves as the methodological base to further investigate the other questions. In the present study, I experiment with a number of different representations of stories, which all have potential advantages and disadvantages for studying story transmission. In particular, I consider the following representations: (i) motif representations (Chapter \ref{ch:motif-classification}), (ii) semantic representations of the character cast of stories (Chapter \ref{ch:animacy}), (iii) representations based on manual narratological text analysis (Chapter \ref{chp:red-riding-hood}), and (iv) `bag-of-words' representations (Chapter \ref{ch:story-networks}). 

In folktale research, the relations between stories are often investigated by means of motif representations. Motifs, such as those found in the \emph{Motif-Index} by Stith Thompson\autocite{thompson:1955}, are considered to be the primary building blocks of stories and their constellation defines the tale type of a story as categorized in folktale catalogs such as \emph{The Types of International Folktales}\autocite{uther:2004}. The current study wishes to contribute to these folklorists' research endeavors by delivering two research instruments which allow researches to analyze large-scale folktale collections and the relations between folktales in terms of their motifs more efficiently. 

The first research instrument is aimed at making existing collections of folklore motifs more accessible. To this end, more than fifty years after the first edition of Thompson's seminal \emph{Motif-Index of Folk Literature}, I present an online search engine in Chapter \ref{ch:motif-classification}, which is tailored to fully disclose the index digitally. This search engine, called \index{MOMFER}MOMFER, greatly enhances the searchability of the \emph{Motif-Index} and provides exciting new ways to explore the collection. This is enabled by the use of modern techniques from the fields of Natural Language Processing and Information Retrieval. The key feature of the search tool is the way in which it allows users to search the \emph{Motif-Index} for semantic concepts, such as `mythical animals', `mortality', or `emotions'. In the first part of Chapter \ref{ch:motif-classification}, I will explain the motivations for creating the search tool, lay bare its production process, and show in a number of case studies how the search tool can be used to explore the index in innovative ways. The second part of Chapter \ref{ch:motif-classification} presents an automated motif identification system. This second research instrument enables scholars to efficiently identify motifs in large-scale folktale collections as well as the relations between stories in terms of their motif representations.

Chapters \ref{ch:animacy}, \ref{chp:red-riding-hood}, and \ref{ch:story-networks} more directly address the evolutionary questions of this study. Chapter \ref{ch:animacy} aims to further our knowledge about how to explain prevalent culture, in which cultural artifacts are differentially preferred and some artifacts are more likely to survive than others. As a case study, I systematically examine the cultural successfulness of fairy tales from the Brothers Grimm collection \emph{Kinder- und Hausmärchen} from a \emph{content}-based perspective (research question 2). I address of which story elements contribute to a story's popularity (or, in other words, which story elements form attractors causing a fairy tale to `stick' and gain popularity). In particular, I look into the question whether the type of characters in a story correlates with its successfulness, i.e.\ whether a character type bias is at play in story selection. An essential methodological prerequisite for addressing this question is to find a way to list the character cast of a story. To this end, I first present a linguistically uninformed computational model for animacy classification with which characters can be automatically identified in folktales. I compare the model to a number of linguistically \emph{informed} models that use features such as dependency tags, and show competitive results. The animacy classifier serves as the methodological base to further investigate the question of whether the cultural success of fairy tales can be (partially) ascribed to character type biases. I apply the animacy classifier to a large collection of Dutch folktales, to develop a typology of character types. By mapping the character casts of a selection of successful and unsuccessful fairy tales from \emph{Kinder- und Hausmärchen} to this typology, I provide empirical evidence for several character types that discriminate between successful and unsuccessful tales.

The remaining two chapters (\ref{chp:red-riding-hood} and \ref{ch:story-networks}) focus on socially informed, \emph{context}-based processes that underlie story transmission and selection. Chapter \ref{chp:red-riding-hood} aims to enhance the understanding of the processes through which stories for children are retold. The study addresses the third research question of this thesis and investigates whether the diachronic development of the world's most famous fairy tale ``Red Riding Hood'' is to be characterized as a process of gradual accumulation of modification, with new variants of the story most likely deriving from previous versions in temporal proximity. Results suggest that new Dutch versions of ``Red Riding Hood'' are potentially subjected to age-dependent selection processes (i.e.\ fashion trends or fads), which means that `young' story versions are preferred in producing a new version. The analysis presented here is based on a digitized collection of Dutch ``Red Riding Hood'' retellings\autocite{folgert_karsdorp_2016_51588}. This collection forms the largest longitudinal collection of children's stories today and holds a wealth of information for studies in (children's) literature, history, sociology, folklore, linguistics, and cultural evolution. 

Finally, further building upon insights and experimental results from Chapter \ref{chp:red-riding-hood}, I pursue a more detailed understanding of the processes through which stories are retold in Chapter \ref{ch:story-networks} (research question 4). A collection of story retellings can be considered as a network of stories, in which links between stories represent pre-textual (or ancestral) relationships. In this chapter, I provide a mechanistic understanding of the structure and evolution of such story networks. I construct story networks for two diachronic story collections: (i) the collection of Dutch ``Red Riding Hood'' retellings from Chapter \ref{chp:red-riding-hood} and (ii) a corpus of paper chain letters. The results confirm and strengthen the analysis presented in Chapter \ref{chp:red-riding-hood} by showing that the formation of these story networks is guided by age-dependent selection processes. Subsequently, I systematically compare these findings with and among predictions of various formal models of network growth to determine more precisely which kinds of attractiveness are also at play or might even be preferred as explicatory models. Carefully studying the structure and evolution of the two story networks, I show that existing stories are differentially preferred to function as a new version's pre-text given three types of attractiveness: (i) frequency-based and (ii) model-based attractiveness which (iii) decays over time.

Earlier versions of parts of this thesis have appeared as journal articles and papers in conference proceedings. Although much of this material has been substantively extended and revised for the current study, I feel that a study about story transmission should at least make its sources explicit:

\begin{refsection}
\nocite{karsdorp:2012,karsdorp:2012b,karsdorp:2013,Karsdorp:2013tk,karsdorp:2015a,karsdorp:2015b,karsdorp:story-networks,folgert_karsdorp_2016_51588} % karsdorp:retelling
\printbibliography[heading=none, keyword=mywork]
\end{refsection}