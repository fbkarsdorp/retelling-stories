%!TEX root = ../main.tex

\chapter*{Acknowledgments}
\addcontentsline{toc}{chapter}{Acknowledgments} \markboth{ACKNOWLEDGMENTS}{} \markright{ACKNOWLEDGMENTS}

\noindent Any story is shaped by the stories that surround it. This dissertation is no different. It could not have been told without the stories of all the wonderful people that have supported me during the past four and a half years. 

First and foremost, I would like to thank my supervisor Prof.\ dr.\ Antal van den Bosch. During my MA training, I read Antal's book on memory-based learning, which was a paradigm-shifting experience. It sparkled my interest in pursuing a Ph.D with a component in language technology. I was very fortunate to be offered a Ph.D position in the Tunes \& Tales project, which not only allowed me to expand my knowledge about computational language modeling, but also to learn it from the best. Antal has my deepest gratitude for all his guidance, intellectual encouragement and much-needed advice on presentations, articles, grant proposals, job positions, and different revisions of the manuscript. 

I thank my second supervisor Prof.\ dr.\ Franciska de Jong for shepherding the manuscript through its final stages. Furthermore, I am grateful to my two co-promotors, Prof.\ dr.\ Theo Meder and Dr.\ Peter van Kranenburg, from whom I benefited immeasurably. Both Theo's profound knowledge of folktales and oral culture, and Peter's invaluable suggestions at numerous occasions have improved the quality of this dissertation considerably. 

A number of colleagues at the Meertens Instituut have been generous with their time and expertise: Berit Janssen, Marianne van Zuijlen, Dong Nguyen, Dolf Trieschnigg, and Iwe Muiser. The embedding of the Tunes \& Tales project within the eHumanities group of the Royal Dutch Academy of Arts and Sciences has given me the opportunity to engage in interesting discussions regarding the emerging field of Digital Humanities, and I wish to thank Prof.\ dr.\ Sally Wyatt, Dr.\ Andrea Scharnhorst, Jeannette Haagsma, and Anja de Haas-Gruys for all their effort and time. Finally, I thank the LAMA's of the Radboud University for encouraging me to leave Amsterdam once in a while.

I am enduringly grateful to my friends and family for their emotional support, patience and  encouragement. I single out three close friends, whom I owe special thanks. First, I thank Basten Stokhuyzen, who implemented the front-end of MOMFER, designed the wonderful cover of this dissertation, and proved to be a steady friend. I am also greatly indebted to Marten van der Meulen, who has been been my research assistant for approximately two years, has co-authored and co-presented several of my articles, and is one of the smartest and funniest persons I know. This dissertation has benefited repeatedly from Marten's wise and thoughtful suggestions, as well as from his enduring patience regarding my endless stream of trial projects. Finally, I thank Lauren Fonteyn, who has been an incisive reader and patient editor of different drafts of this dissertation. Without Lauren's intellectual breadth, critical judgment, and inspiring companionship, I doubt whether this dissertation had been completed at all.