%!TEX root = ../main.tex

\chapter*{Summary}\markboth{SUMMARY}{} \markright{SUMMARY}

\refstepcounter{chapter}
\noindent \emph{Retelling Stories} is a study about the mechanisms underlying story transmission and selection. A central question of the study is whether story transmission can be understood as -- and hence should be described as -- a cultural evolutionary process, and which evolutionary mechanisms can be identified in story transmission. To address this question, the approach presented in this study draws inspiration from research on computational models of cultural evolution~\autocite[E.g.][]{sforzafeldman:1981,boyd_richerson:1985,mesoudi:2011,mesoudi:2015}. By positioning itself explicitly and extensively in dialog with this cultural evolution research program, the current study aims to advance our understanding of story transmission processes by formally and quantitatively characterizing diachronic story developments. While its central focus is to provide a computational-evolutionary approach to story transmission (and encompassing methodological challenges), the study also builds on insights gained from both folkloristic and literary accounts of story transmission\autocite[E.g.][]{stephens_mccallum,boyd:2009,boyd:2010,zipes:2006,zipes:2012,geerts:2014}. As such, this study contributes to the further synthesis of the disciplines of cultural evolution, literary theory and folklore. After presenting the methodological framework and problem statement in Chapter \ref{ch:introduction}, the subsequent four chapters address a number of research questions regarding both methodological and theoretical issues of story transmission.

The first research question constitutes the methodological base of the study and addresses the question of how stories should be formally represented in order to study real-world story transmission with computational means. Chapters \ref{ch:motif-classification} to \ref{ch:story-networks} present four different data representations, each of which has potential (dis)advantages for studying particular aspects of story transmission. In Chapter \ref{ch:motif-classification}, a `classical' representation is employed, which is based on the seminal \emph{Motif-Index} developed by Stith Thompson~\autocite{thompson:1955}. In this representation, motifs are considered to be the basic building blocks of stories, and their constellation defines the story type of a story (e.g.\ story types from \emph{The Types of International Folktales}\autocite{uther:2004}). This story representation, in which stories are primarily considered to be mere instances of story types, contrasts sharply with the data-driven story representations proposed in chapters \ref{ch:animacy} to \ref{ch:story-networks}. These chapters present three different story representations, in which the object of interest is not the overarching story type, but lies primarily with the tale `token', i.e.\ the actual story. Crucially, the main methodological and theoretical advantage of these data-driven representations is that what is considered to be a similarity between stories is never static, but depends on the perspective one wishes to take or the image one aims to construct.

Using the exemplar-based story representations, chapters \ref{ch:animacy} to \ref{ch:story-networks} address a number of questions concerning mechanisms underlying story transmission. Chapter \ref{ch:animacy} contributes to answering the fundamental question of how to explain `contagious' culture~\autocite[Cf.][]{sperber:1996} by investigating which stories `stick' and what \emph{content}-based factors determine their successfulness (Research question 2). Taking the famous fairy tale collection \emph{Kinder- und Hausmärchen} as a case, empirical evidence was given for the existence of a number of character types that affect the popularity of a story. Successful, sticky stories exhibit a significant preference for, for instance, characters referring to family relationships and `minimally counterintuitive' agents~\autocite[See e.g.][]{Barrett:2009}. Unsuccessful stories, by contrast, typically revolve around religious characters and generic groups with rather negative connotations.

Subsequent to this chapter on content-based factors of (un)successful story transmission, the remaining two chapters (\ref{chp:red-riding-hood} and \ref{ch:story-networks}) focus on socially informed, \emph{context}-based processes that underlie story transmission. The goal of Chapter \ref{chp:red-riding-hood} is to enhance our understanding of the processes through which children's stories are retold. The chapter aims to answer the third research question of this thesis and investigates whether the diachronic development of ``Red Riding Hood'' in the Netherlands can be characterized as an (Darwinian) evolutionary process. It is shown that the evolution of the story can be characterized as a `gradual accumulation of modification' in which new versions of ``Red Riding Hood'' tend to modify and adapt prior retellings, and these prior retellings are published in close temporal proximity to these new versions (Research question 3). 

The observed disposition of authors for temporally proximate stories to produce new retellings can be interpreted as an `age bias' in story transmission, comparable to fashion trends or fads. Yet, although the presence of age-dependent selection mechanisms adds important knowledge to our understanding of story transmission processes, it cannot explain why certain story versions of approximately the same age are differentially preferred to function as a story's pre-text. To address this issue, Chapter \ref{ch:story-networks} further scrutinizes the age-dependent selection mechanism while testing and evaluating other biases that could serve as explanatory models for story transmission (Research question 4). On the basis of an extensive and systematic comparison between the output of computational simulations and empirical observations of story transmission, it is shown that the differential selection of stories is affected by the interplay of three biases: (i) a positive frequency bias (in which retellers prefer frequently retold stories), (ii) a model-based bias (e.g. author prestige), and (iii) the aforementioned age bias. 
