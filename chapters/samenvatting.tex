%!TEX root = ../main.tex
%!TEX spellcheck = nl-NL

\chapter*{Samenvatting}\markboth{SAMENVATTING}{} \markright{SAMENVATTING}

\refstepcounter{chapter}
\begin{otherlanguage}{dutch}
\noindent \emph{Retelling Stories} is een onderzoek naar mechanismes die ten grondslag liggen aan de verspreiding en verandering van (volks)verhalen. Een van de centrale vragen in dit onderzoek is of verhaaltransmissie begrepen kan worden als een cultureel evolutionair process en welke evolutionaire mechanismen daarbij een rol spelen. Om deze vraag te beantwoorden, zoekt deze studie aansluiting bij onderzoek naar computationele modellen van culturele verandering~\autocite[Zie bijvoorbeeld:][]{sforzafeldman:1981,boyd_richerson:1985,mesoudi:2011,mesoudi:2015}. Door gebruik te maken van deze computationele benadering van culturele verandering, probeert de huidige studie om diachrone verhaalontwikkelingen op een kwantitatieve en formele manier te bestuderen en te karakteriseren. Hoewel een computationeel-evolutionaire benadering van verhaaltransmissie (en de daarbijhorende methodologische vraagstukken) centraal staat, gaat deze studie ook de dialoog aan met \emph{kwalitatieve} benaderingen van verhaaltransmissie~\autocite[Zie bijvoorbeeld:][]{stephens_mccallum,boyd:2009,boyd:2010,zipes:2006,zipes:2012,geerts:2014} met als doel de synthese tussen culturele evolutie, (jeugd)literatuur en verhaalonderzoek te versterken. Na een inleidend hoofdstuk, waarin het methodologische kader en de probleemstelling van de studie worden gepresenteerd, wordt in de daaropvolgende vier hoofdstukken een evenredig aantal onderzoeksvragen behandeld, waarvan de onderzoeksrelevantie zowel methodologisch als theoretisch van aard is.  

De eerste onderzoeksvraag vormt het methodologische fundament van de studie en richt zich op de representatie en formalisatie van verhalen om historische verhaaltransmissie en -selectie computationeel te kunnen bestuderen. In hoofdstuk \ref{ch:motif-classification} tot en met \ref{ch:story-networks} worden vier verschillende datarepresentaties onderzocht, die elk potentiële voor- en nadelen hebben voor de bestudering van bepaalde aspecten van verhaaltransmissie. Hoofdstuk \ref{ch:motif-classification} presenteert een `klassieke' representatie van verhalen die gebaseerd is op de beroemde \emph{Motif-Index} ontwikkeld door Stith Thompson\autocite{thompson:1955}. In deze representatie worden motieven beschouwd als de primaire bouwstenen van verhalen en hun verzameling bepaalt tot welk verhaaltype een verhaal behoort (zoals de verhaaltypes in \emph{The Types of International Folktales}\autocite{uther:2004}). Deze representatie, waarin verhalen primair worden beschouwd als instantiaties van bepaalde verhaaltypes, staat haaks op de data-gedreven representaties van verhalen in de hoofdstukken \ref{ch:animacy} tot en met \ref{ch:story-networks}. In deze hoofdstukken worden drie representaties voorgesteld waarin niet het verhaaltype het primaire onderwerp van studie is, maar het eigenlijke verhaal of `verhaal\emph{token}'. Een belangrijk voordeel van deze token-gebaseerde representaties ten opzichte van de `klassieke', type-gebaseerde representatie, is dat overeenkomsten tussen verhalen niet statisch zijn, maar afhankelijk van het door de onderzoeker gekozen perspectief. 

In hoofdstuk \ref{ch:animacy} tot en met \ref{ch:story-networks} worden de data-gedreven verhaalrepresentaties gebruikt om een aantal onderzoeksvragen te beantwoorden over mechanismes van verhaaltransmissie. Hoofdstuk \ref{ch:animacy} levert een bijdrage aan het beantwoorden van de fundamentele vraag hoe `besmettelijke' cultuur verklaard kan worden\autocite[Vergelijk:][]{sperber:1996} door te onderzoeken welke type(s) verhalen beklijven en welke \emph{verhaalinhoudelijke} factoren hierop van invloed zijn (onderzoeksvraag 2). De resultaten laten zien dat de (im)populariteit van sprookjes in sterke mate beïnvloed wordt door het type personages dat in de verhalen voorkomt. Zo hebben populaire, beklijvende verhalen een significante voorkeur voor personages uit de familiesfeer en personages die `minimaal tegenintuïtief' genoemd kunnen worden~\autocite[Zie bijvoorbeeld][]{Barrett:2009}. Impopulaire verhalen, daarentegen, kenmerken zich door bijvoorbeeld het opvoeren van generieke (vaak negatief geëvalueerde) personagegroepen en personages uit de religieuze sfeer.

Na dit hoofdstuk over verhaalinhoudelijke factoren van (on)\-succesvolle verhaaltransmissie volgen twee hoofdstukken over sociaal-geïnformeerde, \emph{context-gebaseerde} processen die ten grondslag liggen aan verhaaltransmissie. Het doel van hoofdstuk \ref{chp:red-riding-hood} is om een beter begrip te ontwikkelen over de processen waarmee kinderverhalen worden herverteld. Het hoofdstuk probeert een antwoord te geven op de derde onderzoeksvraag van deze dissertatie en onderzoekt of de diachrone ontwikkeling van ``Roodkapje'' in het Nederlandse taalgebied gekarakteriseerd kan worden als een (Darwiniaans) evolutieproces, waarin aanpassingen van het verhaal geleidelijk opeengestapeld worden en nieuwe varianten van het verhaal afgeleid zijn van versies uit het recente verleden. Uit de resultaten blijkt dat nieuwe Nederlandse hervertellingen van ``Roodkapje'' inderdaad onderworpen zijn aan leeftijdsafhankelijke selectieprocessen (vergelijkbaar met modetrends). Dat betekent `jonge' vertellingen de voorkeur krijgen om een nieuwe vertelling op te baseren.

Hoewel de leeftijdsafhankelijke selectiemechanismen voor ``Roodkapje'' belangrijke inzichten toevoegen aan onze kennis over verhaaltransmissie, biedt het nog geen verklaring voor de vraag waarom specifieke verhaalversies worden verkozen boven andere versies van ongeveer dezelfde leeftijd om een nieuwe hervertelling op te baseren. Om deze vraag te onderzoeken, neemt hoofdstuk \ref{ch:story-networks} de leeftijdsafhankelijke selectievoorkeur verder onder de loep en onderzoekt daarnaast welke andere selectievoorkeuren een rol spelen in verhaaltransmissie (onderzoeksvraag 4). De resultaten van een systematische vergelijking tussen computationele simulaties met empirische observaties van verhaaltransmissie laten zien dat de differentiële selectie van verhalen sterk beïnvloed wordt door de interactie van een drietal factoren: (i) een frequentievoorkeur (waarbij hervertellers kiezen voor populaire verhalen), (ii) een `modelvoorkeur' (waarbij hervertellers zich laten leiden door bijvoorbeeld de prestige van een auteur of uitgave) en (iii) de eerder besproken leeftijdsvoorkeur. 

\end{otherlanguage}