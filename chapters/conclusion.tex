%!TEX root = ../main.tex

\chapter{General Discussion}\label{ch:conclusion}
\markboth{GENERAL DISCUSSION}{} \markright{GENERAL DISCUSSION}

\chapterprecishere{``[Darwin] is no longer the authoritative old man with a beard substituting for God.''}{Gillian Beer}{Darwin's Plots}

\vspace{3.5ex plus 1ex minus .2ex}
\noindent In what preceded, I have offered new perspectives on the mechanisms underlying story transmission and selection. In essence, the approach presented here builds on the insights gained from both folkloristic and literary accounts of story transmission\autocite{stephens_mccallum,boyd:2009,boyd:2010,zipes:2006,zipes:2012,geerts:2014}. However, while such accounts have undoubtedly yielded a wealth of insightful ideas about the mechanisms at play in story transmission, their arguments and claims often remain programmatic and are based on informal verbal arguments that do not allow for rigorous quantitative evaluations. Recent developments in the cultural evolution research program have shown the benefits of employing \emph{formal} models to further the understanding of cultural selection processes. By positioning itself explicitly and extensively in dialog with the cultural evolution research program\autocite{sforzafeldman:1981,boyd_richerson:1985,mesoudi:2011,mesoudi:2015}, the present study aimed to advance our understanding of story transmission processes by means of such formal models of cultural evolution. The added value of the computational approach to story transmission presented in this study is that it yields specific, replicable predictions that can be quantitatively and rigorously evaluated against real-world data. At the same time, it also enables us to isolate and systematically compare forces of selection at play in story transmission. In this concluding chapter, I will synthesize the findings presented in Chapters \ref{ch:motif-classification} to \ref{ch:story-networks} from (i) a methodological perspective that assesses various ways to formally represent stories in order to study real-world story transmission (Research question 1), and (ii) a theorizing perspective that evaluates what the preceding analyses tell us about the factors that determine story transmission and selection (Research question 2--4).

\section[methodological challenges]{Methodological challenges}

While the cultural evolution research program has investigated the mechanisms of story transmission in laboratory contexts\autocite{mesoudiwhiten:2008}, few attempts have been made to apply this framework and encompassing research methods to real-world, historical story data\autocite[Notable exceptions are:][]{tehrani:2013,daSilva:2016}. Therefore, the present study needed to resolve a number of methodological challenges prior to addressing its main research questions. Perhaps the most important methodological issue, which is addressed throughout this entire study, is how stories should be represented in order to computationally study real-world story transmission and selection (Research question 1). As there is no simple `one solution fixes all' answer to this question, I explored a number of different data representations, each of which has potential advantages and disadvantages for studying certain aspects of story transmission. 

In traditional folktale research, the methods used to describe and study relations between stories revolves primarily around the motifs from Thompson's \emph{Motif-Index}\autocite{thompson:1955}. In such studies, motifs are used as the primary descriptive units of stories, and their constellation defines their corresponding tale type as described in, for example, \citeauthor{uther:2004}'s tale type catalog\autocite{uther:2004}. The problem with these traditional approaches is that, in practice, the motifs are not used to identify tale types. Rather, the typical classification strategy in such analyses is an \emph{a priori} assigning of the story under investigation to a single tale type, before any of its motifs are properly identified. As a consequence, the object of investigation is considered solely in light of the motifs that are associated with that particular type. Thus, the classification strategy of predefined tale types and encompassing motifs \emph{encourages} to foreclose particular connections between stories. Neither the concept of a tale type, nor the motifs that supposedly constitute them, is without risk. 

Another problem with tale types and predefined motifs is that they inevitably decontextualize stories. As Marina Warner aptly puts it: the folktale type catalog ``provides a list of ingredients [i.e.\ motifs] and recipes [i.e.\ tale types] with no evocation of their taste or the pleasure of the final dish, nor sense of how or why it was eaten''\autocite[XVIII]{warner:1995}. In other words, stories are torn from their original contexts and might be generalized to a level too abstract to retain any significance. If we, for instance, would analyze the more than four hundred versions of ``Red Riding Hood'' from Chapter \ref{chp:red-riding-hood} in terms of the presence or absence of the motifs listed under tale type ATU 333 ``Little Red Riding Hood'', most versions would become indistinguishable. As such, the motif classification leaves us without means to explain the observed variation and progression through time.

One of the central claims in this study is that tale types should serve the interpretation of actual stories; not the other way around. Tale types can in fact be dangerous when the classification of stories into tale types becomes a static, authoritative, `this or that' enterprise that mutes the many consonant and dissonant resonances with other stories\autocite[Cf.][]{frank:2010}. The potential danger of foregrounding types instead of stories is amplified by Arthur Frank in the context of stories describing experiences of illness:
\begin{quote}
  \noindent ``Typologies risk putting stories in boxes, thus allowing and even encouraging the monological stance that the boxes are more real than the stories, and the types are all that need to be known about the stories. In a world where simplification is a pretext for knowing, and knowing is a pretext for controlling, typologies are risky.''\autocite[118--119]{frank:2010}
\end{quote}
In principle, there is nothing wrong with using tale types as analytical tools, as long as we recognize them to be only one out of many possible perspectives. Typologies invite us to make connections among similar stories, which helps researchers to get a grip on stories and enhance their interpretation. However, when the identification of types becomes a goal in itself, we run the risk of remaining blind to the variation exposed within a category, and of foreclosing both existing and potential relations between stories belonging to separate categories. This risk is anything but trivial if we want to explain how stories are created, interpreted, adapted, and retold. The key or Bakhtinian ``dialogical trick'', as \citeauthor{frank:2010} suggests, is to sustain \emph{openness} and ``[t]ypologies should never be considered final''\autocite[119, 121]{frank:2010} nor should the understanding of which stories fit what tale type.

In contrast with arborescent classification systems which consider tale types to be primary, I advocated an exemplar-based approach of stories. Exemplar-based approaches differ from typology-based approaches in that the object of interest is not the overarching tale type, but lies primarily with the tale `token', i.e.\ the actual story\autocite[119]{frank:2010}. Each story is considered to be a unique entity that is related to other story exemplars in various -- not necessarily predetermined -- ways. In Chapter \ref{ch:story-networks}, I conceptualize the relationships between story exemplars as a network that consists of more and less densely connected clusters of stories. As such, tale types can be considered to be emerging implicitly from highly dense clusters of similar stories. What is considered to be a similarity between stories in the present approach is never static, but depends on the perspective one wishes to take or the image one aims to construct. In Chapter \ref{ch:story-networks}, for example, similarities between stories were determined on the basis of their vocabulary (using a bag-of-words representation), whereas in the study about ``Red Riding Hood'' in Chapter \ref{chp:red-riding-hood}, more abstract perspectives such as `time', `plot' and `irony' contributed to the conceived similarities. Chapter \ref{ch:animacy} presented yet another perspective. To study the factors determining the successfulness of fairy tales from the Brothers Grimm's \emph{Kinder- und Hausmärchen}, I focused on the stories' character cast, which were represented as semantic vectors (i.e.\ word embeddings). In studying story transmission and selection from an evolutionary perspective, it is of utmost importance to acknowledge that relations between stories are malleable and can only be accounted for if we adopt data-driven representations of stories that do not superimpose predefined categories onto stories.

Each of the data-driven representations explored in the present study have certain advantages and disadvantages. First, the character representations employed in Chapter \ref{ch:animacy} have the advantage of conveying semantic information. In addition, these representations require minimal manual labor, which greatly facilitates analyses of large-scale data collections. However, these representations only represent stories on a single dimension (i.e.\ the character cast), and ignore the fact that there are many other dimensions that are (potentially) of equal importance in establishing connections between stories. The bag-of-words representations used in Chapter \ref{ch:story-networks}, by contrast, represent stories on \emph{multiple} dimensions. Bag-of-words representations are powerful, widely used representations, which are computationally efficient and -- even more so than the character representations in Chapter \ref{ch:animacy} -- require minimal manual labor. Despite the fact that bag-of-words make the rather crude simplification of ignoring many higher-order aspects of texts (e.g.\ word order, sentences, structure), they provide effective means to expose important relationships between stories. 

The bag-of-words representations of Chapter \ref{ch:story-networks} contrast sharply with the detailed and fine-grained representations employed in Chapter \ref{chp:red-riding-hood}. The representations in Chapter \ref{chp:red-riding-hood} depend on a rigorous and extensive narratological text analysis consisting of over three hundred questions, which served as the basis for studying the evolution of ``Red Riding Hood'' in a large corpus of Dutch retellings. This questionnaire consists of various questions concerning high- and low-level aspects of the story, ranging from the way characters are named to aspects of theory of mind. While the bag-of-words representations of Chapter \ref{ch:story-networks} also represent stories in a high-dimensional space, the questionnaire approach has the advantage of enabling researchers to represent stories on multiple dimensions with varying degrees of abstraction and detail. The obvious drawback of these representations is, however, that they require labor-intensive and subjective manual analysis (both in the choice for particular questions as well as in the answers given to these questions).

Note that many of the questions employed in Chapter \ref{chp:red-riding-hood} resemble motifs from \citeauthor{thompson:1955}'s \emph{Motif-Index of Folk Literature}\autocite{thompson:1955}, which formed the central representation form of stories in Chapter \ref{ch:motif-classification}. Questions such as ``Is the wolf's belly filled with stones or some other material?''\ or ``Is Red Riding Hood eaten by the wolf?''\ can be linked to, respectively, motifs Q426 (\emph{Wolf cut open and filled with stones as punishment.}) and K2011 (\emph{Wolf poses as ``grandmother'' and kills child.}). Yet, the fact that there are some obvious similarities between some of Thompson's motifs listed under tale type ATU 0333 and some parts of the questionnaire does not imply that the list of motifs and the questionnaire should be considered to be on a par. The questionnaire is to be considered as a data-driven superset of tale type ATU 333 and its constituting motifs. That is to say, the questionnaire entails the given motifs and, at the same time, adds numerous other dimensions of variation on which the stories can be compared. Importantly, these dimensions of comparison (i) arise from the collection of stories under investigation rather than being imposed onto the data and (ii) are never finalized but remain open to reconfiguration and addition when new versions of the story are added to the collection.

\section{Story Transmission and Theory}

Using these data-driven representations, I addressed a number of questions concerning the mechanisms underlying story transmission and selection in Chapters \ref{ch:animacy} to \ref{ch:story-networks}. Chapter \ref{ch:animacy} presented original empirical work that contributes to answering Sperber's fundamental question of how to explain `contagious' culture\autocite{sperber:1996}, or, more specifically, which stories successfully `stick' and what content-based factors determine their successfulness. In particular, I tested the hypothesis of whether a `character bias' (i.e.\ a disposition for particular character types) is at play in the cultural selection of fairy tales (Research question 2). Taking the famous fairy tale collection \emph{Kinder- und Hausmärchen} (1857) by the Brothers Grimm as a case study, I provided empirical evidence for the existence of several character types that affect the successfulness of a story. It was shown that successful fairy tales exhibit a significant preference for (i) characters with names that refer to family relationships, (ii) animal characters and (iii) minimally counterintuitive agents. 

It is interesting to regard these findings in light of previous proposals with respect to the role of characters in story transmission. Recall the epigraph of Chapter \ref{ch:motif-classification}, which recites \citeauthor{thompson:1977} stating that a motif is ``the smallest element in a tale having a power to persist in tradition''.\autocite[415]{thompson:1977} \citeauthor{thompson:1955}'s motifs generally fall into three categories, the first being the tale's characters. These characters, in order to have the power of persistence Thompson attributes to motifs, must exhibit something ``unusual and striking''\autocite[415]{thompson:1977}. Interestingly, these properties typically apply to marvelous creatures such as witches, ogres and fairies, or, in other words, the so-called minimally counterintuitive agents (cf.\ Chapter \ref{ch:animacy}). The present study provides empirical evidence for \citeauthor{thompson:1977}'s hypothesis that such marvelous characters have a power to persist in tradition. At the same time, however, the present study updates Thompson's proposal by showing that minimally counterintuitive agents are not only persisting motifs, but also serve as popularizing \emph{attractors} in the cultural selection of fairy tales. Moreover, the empirical findings of the current study add further support to hypothesized content-based biases in story transmission (e.g.\ a disposition for social information\autocite{reysen:2011,Stubbersfield:2014}). Although these biases have to some extent been investigated through experimental transmission chain studies in the lab\autocite[See, for instance,][]{mesoudi:2006,barrett:2004,Upal:2007,HarmonVukic:2009,Barrett:2009}, the current study presents a unique account in support of these hypotheses on the basis of historical, real-world data. 

Importantly, while current experimental research and research in literary and folkloristic studies primarily focuses on detecting positive biases in story transmission (i.e.\ traits that accrue a story's chances of success), the concept of `negative biases' has been largely overlooked. To address this hiatus in our understanding of story transmission, the current study homed in on the so-called `impopularizing' character types of fairy tales. The results suggested that unsuccessful fairy tales from \emph{Kinder- und Hausmärchen} typically revolve around (i) religious characters, (ii) criminals and hooligans and (iii) generic groups with a rather negative connotation. This negative bias away from stories with generic, indefinite groups ties in with the study on the evolution of ``Red Riding Hood'' (Chapter \ref{chp:red-riding-hood}), in which the wolf has become less generic and more individualized over time. Thus, it would be interesting to further pursue the question of whether these findings are part of a more general development in which individualized stories exhibit a transmission advantage in future research.

The progressive individualization of the wolf is only one of the many transformations that ``Red Riding Hood'' has undergone in the past centuries. While originally intended as a parable intended to warn young ladies of the French court about debonair and sweet-talking rapacious wolfs, ``Red Riding Hood'' has become an increasingly autonomous story and has been subjected to experimentation and reconfiguration. In Chapter \ref{chp:red-riding-hood}, I have systematically and quantitatively investigated this development on the basis of a large longitudinal collection of Dutch retellings of ``Red Riding Hood''. I demonstrated that the development of the story can be characterized as a gradual accumulation of modification: new versions of a story tend to modify and adapt prior retellings, and these prior retellings are published in close temporal proximity to these new versions (Research question 3). The resulting diachronic picture resembles a chain of retellings, in which retellers introduce adaptations and innovations. If these adaptations and innovations are further retold, they may come to gradually replace existing story elements. As such, I have argued, the progressive alteration of ``Red Riding Hood'' in the Netherlands can be interpreted as a cultural evolutionary process.

Finally, the observed preference of authors to produce new versions of ``Red Riding Hood'' on the basis of temporally proximate versions can be interpreted as evidence for a second bias in story transmission (i.e.\ `age bias'). Yet, although this age-dependent selection mechanism for ``Red Riding Hood'' adds important insights to our knowledge about story transmission, it cannot explain why particular story versions versions of approximately the same age are differentially preferred to function as pre-text for new retellings. To address these issues, Chapter \ref{ch:story-networks} further scrutinized this age bias while testing other biases that might function as explanatory models in story transmission (Research question 4). By extensively and systematically comparing outcomes of computational simulations with real-world observations of story transmission, it was shown that story transmission is affected by a positive frequency bias as well as a model-based bias which reduces in strength over time.

The simulations of network growth employed in Chapter \ref{ch:story-networks} represent simplified mathematical models of story transmission. A major theoretical benefit of employing such simulation models is that they call for detailed and replicable definitions. As such, simulation models force researchers to make their theoretical assumptions explicit. Moreover, simulation models also exhibit an interesting degree of simplicity. In the social sciences, the use of such simplified simulation models has often been fiercely criticized for their inability to capture the complex nature of cultural phenomena\autocite{mesoudi:2011}. Yet, I wish to underscore that -- as already pointed out by Mesoudi -- ``the fact that simulations are highly simplified is the very reason for their usefulness''\autocite[326]{mesoudi:2005}. By deliberately keeping the models simple, researchers can easily isolate and manipulate single variables under highly controlled conditions\autocite{mesoudi:2005,mesoudi:2011}. In recent years, the use of evolutionary models has had revolutionary effects across the social sciences\autocite{mesoudi:2015}, and I believe that the application of such models could also lead to a wealth of new insights in folkloristic and literary studies.

\section{An outlook to future research}

Yet, despite the attested descriptive successes we should refrain from rejoicing too much in technological positivism. Comprehensive, in-depth study of the phenomena observed in real-word story transmission processes still faces a number of challenges. Perhaps the biggest challenge is to develop methods to address the relevant questions considering that there is a lack of data. Longitudinal data collections such as the ``Red Riding Hood'' corpus presented in Chapter \ref{chp:red-riding-hood} are virtually non-existent. In an ideal-case scenario, we would be in possession of data collections that specify the exact paths of transmission between individuals and generations, i.e.\ who learns or copies from whom\autocite{kandler:2015}. The rare existing data collections, however, merely represent sparse, incomplete samples of the actual data and none of them provide explicit information about the transmission paths. An additional complication to the matter is that the currently available data only allow us to study the evolution of a small number of individual stories. Making more data available to obtain a more profound understanding of story selection dynamics that transcend those of the individual story would, however, be time-consuming and laborious, and the effort becomes even more impossible to accomplish if we want to broaden the perspective to the comparison of story transmission processes between different cultures.

Given this lack of diachronic data, I consider it to be of crucial importance that we continue to explore different and new methodologies to analyze real-world story transmission. As pointed out by \citeauthor{kandler:2015}, fine-grained individual-level data of cultural change are difficult to obtain, yet many archaeological and anthropological data collections do describe aggregate, population-level outcomes of evolutionary processes.\autocite{kandler:2015} Such outcomes take the form of frequency distributions of variants of a cultural trait at a particular moment in time, or of frequency changes of variants over time. A central question in the cultural evolution program is how such outcomes can be used to explain observed periods of cultural change\autocite{Mesoudi:2009,kandler:2015,kandler:2015b}. For story transmission processes, good examples of population-level outcomes are popularity polls, indices of reprints and bibliographical databases\autocite{koman:2008,joosen:2014}. All of this information is much easier to obtain than large-scale longitudinal full-text corpora, providing us with a useful starting point for addressing the above-mentioned issues. I take it to be an intriguing question how such sparse, population-level data can be used to further identify and map out the underlying acting-forces in story transmission and selection processes.

With the present study, I hope to have revealed and underscored the descriptive and theoretical advantages as well as the further potential of adapting and appropriating computational-evolutionary models to story transmission research. Yet, just as the life span of a story depends on its being picked up and further retold, the continuation of the ideas and models presented above depends on its being adapted and appropriated in new retellings within future research on story transmission, cultural evolution, and folkloristic and literary studies. Hopefully, the present `story' will contribute to ``spark related thoughts, responses, and readings''\autocite[160]{sanders:2006} in retellings to come. 
